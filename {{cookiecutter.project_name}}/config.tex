%!TEX root = {{cookiecutter.project_name}}.tex


% Добавьте ссылку на файлы с текстом работы
% Можно использовать команды:
%   \input или \include
% Пример:
%    \input{mainfiles/1-section} или \include{mainfiles/2-section}
% Команда \input позволяет включить текст файла без дополнительной обработки
% Команда \include при включении файла добавляет до него и после него команду
% перехода на новую страницу. Кроме того, она позволяет компилировать каждый файл
% в отдельности, что ускоряет сборку проекта.
% ВАЖНО: команда \include не поддерживает включение файлов, в которых уже содержится команда \include,
% т.е. не возможен рекурсивный вызов \include
\newcommand*{\Source}{
    % Some examples:
    % \include{mainfiles/1-intro}
    % \input{2-section}
}

% Название работы
\newcommand{\Title}{%
    Название статьи%
}

% Имя автора работы
\newcommand{\Author}{%
    Чижов Иван Владимирович%
}

% Информация о годе выполнения работы
\newcommand{\Date}{%
    % 2006%
    \today%     % Текущий год
}

% Аннотация
\newcommand{\Abstract}{%
    Это краткая аннотация.%
    Статья пока ни о чём.%

    Добавьте ей немного смысла.%
}

%%%% Переключите статус документа для отладки
%%%% В режиме draft документ собирается очень быстро
%%%% и выводится полезная информация о том
%%%% какие строки вылезают за границы документа, что удобно для борьбы с ними
\newcommand{\Status}{%
    draft%
    % final%
}
